\section{\textsc{DeepDataFlow} Dataset}

The \textsc{DeepDataFlow} dataset comprises: 461k LLVM-IR files assembled from a
range of sources, \programl representations of each of the IRs, and 15.4M sets
of labeled graphs for the five data flow analyses described in the previous
section, totaling 8.5B classification labels. The dataset is publicly
available~\cite{chris_cummins_2020_4247595}.

\paragraph{Programs} We assembled a 256M-line corpus of real-world LLVM-IRs from
a variety of sources, summarized in Table~\ref{table:corpus}. We selected
popular open source software projects that cover a diverse range of domains and
disciplines, augmented by uncategorized code mined from popular GitHub projects
using the methodology described by~\citet{Cummins2017a}. Our corpus comprises
five source languages (C, C++, Fortran, OpenCL, and Swift) covering a range of
domains from functional to imperative, high-level to accelerators. The software
covers a broad range of disciplines from compilers and operating systems to
traditional benchmarks, machine learning systems, and unclassified code
downloaded from popular open source repositories.

\begin{table*}
  \centering%
  \footnotesize
\begin{tabular}{L{4cm} L{1.45cm} L{2.95cm} r r}
  & Language & Domain & IR files & IR lines\\
  \toprule
  BLAS 3.8.0 & Fortran & Scientific Computing & 300 & 345,613\\
  GitHub & C & Various & 38,109 & 74,230,264 \\
  \multirow{2}{*}{} & OpenCL & \multirow{2}{*}{} & 5,224 & 9,772,858 \\
                  & Swift & & 4,386 & 4,586,161\\
  Linux 4.19 & C & Operating Systems & 13,418 & 41,904,310 \\
  NPB~\citep{Bailey1991a} & C & Benchmarks & 122 & 255,626 \\
  \citet{Cummins2017b} & OpenCL & Benchmarks & 256 & 149,779\\
  OpenCV 3.4.0 & C++ & Computer Vision & 432 & 2,275,466 \\
  POJ-104~\citep{Mou2016} & C++ & Standard Algorithms & 397,032 & 104,762,024 \\
  Tensorflow~\citep{Abadi} & C++ & Machine learning & 1,903 & 18,152,361 \\
  \midrule
  Total & & & 461,182 & 256,434,462 \\
\end{tabular}
%
  \caption{The \textsc{DeepDataFlow} LLVM-IR corpus.}%
  \vspace{.5em}
  \label{table:corpus} %
\end{table*}

\paragraph{\programl Graphs} We implemented \programl construction as an
\texttt{llvm::ModulePass} using LLVM version 10.0.0 and generated a graph
representation of each of the LLVM-IRs. \programl construction takes an average
of 10.72ms per file. Our corpus of unlabeled graphs totals 268M vertices and
485M edges, with an average of 581 vertices and 1,051 edges per graph. The
maximum edge position is 355 (a large \texttt{switch} statement found in a
TensorFlow compute kernel).

\paragraph{Data Flow Labels} We produced labeled graph instances from the
unlabeled corpus by computing ground truth labels for each of the analysis tasks
described in Section~\ref{appendix:dataflow_problems} using a traditional
analysis implementation. For each of the five tasks, and for every unlabeled
graph in the corpus, we produce $n$ labeled graphs by selecting unique source
vertices $v_{0} \in V$, where $n$ is proportional to the size of the graph:
\begin{equation*}
n = \min \left( \left\lceil \frac{|V|}{10} \right\rceil, 10 \right)
\end{equation*}
Each example in the dataset consists of an input graph in which the source
vertex is indicated using the \emph{vertex selector}, and an output graph with
the ground truth labels used for training or for evaluating the accuracy of
model predictions. For every example we produce, we also record the number of
steps that the iterative analysis required to compute the labels. We use this
value to produce subsets of the dataset to test problems of different sizes,
shown in Table~\ref{table:ddf_sizes}.

\begin{table*}
\centering%
\footnotesize
\begin{tabular}{r r r r r r}
  & DDF-30 & DDF-60 & DDF-200 & \ddfinf{} \\
  \toprule
  Max.\ data flow step count & 30 & 60 & 200 & 28,727 \\
  \#.\ classification labels & 6,038,709,880 & 6,758,353,737 & 7,638,510,145 & 8,623,030,254 \\
  \#.\ graphs (3:1:1 train/val/test) & 10,951,533 & 12,354,299 & 13,872,294 & 15,359,619 \\
  Ratio of full test set & 71.3\% & 80.4\% & 90.3\% & 100\% \\
\end{tabular}
\caption{Characterization of \textsc{DeepDataFlow} subsets.}
\label{table:ddf_sizes}
\end{table*}

We divided the datasets randomly using a 3:1:1 ratio for training, validation,
and test instances. The same random allocation of instances was used for each of
the five tasks. Where multiple examples were derived from a single IR, examples
derived from the same IR were allocated to the same split.

As binary classification tasks, data flow analyses display strong class
imbalances as only a small fraction of a program graph is typically relevant to
computing the result set of an analysis. On the \ddfinf{} test sets, an accuracy
of 86.92\% can be achieved by always predicting the negative class. For this
reason we report only binary precision, recall, and F1 scores with respect to
the positive class when reporting model performance on \textsc{DeepDataFlow}
tasks.
